%%%%%%
%
% PROJECT 3 - MODELING OIL PRODUCTION
%
% filename: modeling_oil_production.tex
% last modified: 2014-7-16
%
%%%%%%%
%
%
%%%%%%%

\documentclass
[justified,nohyper]
{tufte-handout}

\usepackage{booktabs}
\usepackage{graphicx}
\usepackage{kmath,kerkis} % The order of the packages matters; kmath changes the default text font
\usepackage[T1]{fontenc}

\begin{document}
\begin{center}
\includegraphics[scale=0.3]{oil.jpg}
\end{center}
\section{Advanced Calculus Project 3: Modeling World Oil Production}
\newthought{There are two things} that are clear about crude oil. One is that we 
use a lot of it. The world consumption of crude oil is approximately 80 million 
barrels per day, and world consumption grew by 3.4\% in 2004.\sidenote{\textit{
New Scientist}, 21 May 2005, page 7.}

The other is that the Earth's oil reserves are finite. The processes that
created crude oil that we use today are fairly well understood. There may be 
significant deposits of crude oil yet to be discovered, but it is a limited 
resource.

Governments, economists, and scientists argue endlessly about almost every other 
aspect of oil production. Exactly how much oil is left in the Earth and what 
fraction of that oil can or will ever be removed is difficult to estimate and 
has significant financial ramifactions. Substantial disagreement on oil policy 
is not surprising.

Predictions of the decline in production are notoriously difficult, and it is 
easy to find examples of such predictions that ended up being absurdly wrong.
\sidenote{For example, http://goo.gl/Yunzsk} On the other hand, sometimes 
predictions of decline in production are accurate. In \textit{Hubbert's Peak}
\sidenote{Deffeyes, K.S., \textit{Hubbert's Peak}, Princeton University Press, 
Princeton and Oxford, 2001.}, Kenneth Deffeyes recounts the work of geologist M. 
King Hubbert. Hubbert fit a logistic model, precisely the type we have been 
looking at for things like population growth, to the production data for crude 
oil in the United States. Using production data up to the mid 1950s along with 
approximations of the total amount of recoverable crude oil, Hubbert predicted 
that production would peak in the U.S. in the 1970s. He was right.

\newthought{In this project} you will model the U.S. and world crude oil 
production using a logistic model, where the carrying capacity represents the 
total possible recoverable crude oil. Your final report should address the 
following items:

\begin{enumerate}
  \item Find parameter values for a logistic differential equation that fit the 
  crude oil production data for the U.S. (see Table \ref{oil1}).\sidenote{Data 
  from Twentieth Century Petroleum Statistics, 1984, by DeGolyer and MacNaughton 
  and http://www.eia.doe.gov} Be sure that you include in your report a clear 
  description of how you determined the parameter values for your logistic 
  differential equation and that you should a graph of the data along with your 
  solution curve.
  \item Predicting both the growth rate and the total amount of recoverable crude 
  oil from the data is difficult. Model the crude oil production of the U.S. 
  assuming the total amount of recoverable crude oil in the U.S. is 200 billion 
  barrels. (This assumption includes what has already been recovered and serves 
  the role of the carrying capacity in the logistic model.)
  \item Repeat Part 2 replacing 200 billion barrels with 300 billion barrels.
  \item Model the world crude oil production based on estimates of total 
  recoverable crude oil (past and future) of 2.1 trillion barrels and 3 trillion 
  barrels. (Both of these estimates are commonly used. They are based on 
  differing assumptions concerning what it means for crude oil to be ``
  recoverable.'') When do the models predict that the rate of production of oil 
  reaches its maximum?
  \item The decline in production of crude oil will certainly result in an 
  increase in price of oil products. This price increase will provide more funds 
  for crude oil production, perhaps slowing the rate of decline. Describe how 
  this price increase might affect the predictions of your model for world oil 
  production and how you might modify your model to reflect these assumptions.
\end{enumerate}

\textbf{Your report:} As always, your report needs an abstract and conclusion 
with sufficient explanation of what you did to solve this problem. Please present 
your models one at a time, with separate explanations of each. Discuss how well 
they fit the data and how sensitive this fit is to small changes in the 
parameters. As the title of this project suggests, you will need to come to some 
conclusion about when the ``peak'' occurs and how long we have until the oil ``
runs out.'' Both of these dates should be clear from reading your report and 
should be supported by your mathematical analysis. 

\begin{fullwidth}
\begin{table}
\begin{tabular}{ccc||ccc}
\toprule
Year & U.S. Oil & World Oil & Year & U.S. Oil & World Oil \\
\midrule
1920-24 & 2.9 & 4.3 & 1970-74 & 17.0 & 93.9 \\
1925-29 & 4.2 & 6.2 & 1975-79 & 15.3 & 107 \\
1930-34 & 4.3 & 7.0 & 1980-84 & 15.8 & 101 \\
1935-39 & 5.8 & 9.6 & 1985-89 & 15.2 & 104 \\
1940-44 & 7.5 & 11.3 & 1990-94 & 12.9 & 110 \\
1945-49 & 9.2 & 15.2 & 1995-99 & 11.5 & 118 \\
1950-54 & 11.2 & 22.4 & 2000-04 & 10.4 & 126 \\
1955-59 & 12.7 & 31.9 & 2005-09 & 9.3 &  130\\
1960-64 & 13.4 & 44.6 & 2010-12 & 6.4 &  80.5 \\
1965-69 & 15.8 & 65.4 &         &     &     \\
\bottomrule
\caption{\label{oil1} Oil production per five year periods in billions of barrels}
\end{tabular}
\end{table}
\end{fullwidth}

\end{document}