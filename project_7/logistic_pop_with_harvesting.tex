%%%%%%
%
% PROJECT 5 - LOGISTIC POPULATION MODELS WITH HARVESTING
%
% filename: logistic_pop_with_harvesting.tex
% last modified: 2014-11-22
%
%%%%%%%
%
% IN THIS PROJECT, STUDENTS HAVE TO MODEL THE POPULATION WITH A LOGISTIC MODEL
% TAKING INTO CONSIDERATION HARVESTING AT DIFFERENT RATES.
%
%
%%%%%%%
%
%
% 
%%%%%%%

\documentclass
[justified,nohyper]
{tufte-handout}

\usepackage{booktabs}
\usepackage{graphicx}
\usepackage{kmath,kerkis} % The order of the packages matters; kmath changes the default text font
\usepackage[T1]{fontenc}

\begin{document}
\includegraphics[scale=0.3]{fish.png}
\section{Advanced Calculus Project 5: Logistic Population Growth with Harvesting}

\newthought{Logistic Models of} population growth can be very useful in modeling population
growth in 
an environment with finite resources. In some cases, we may want to also ``harvest'' part of the 
population at regular intervals. In particular, you should imagine a fish population subject to 
various degrees and types of fishing. The differential equation models for three types of
harvesting are given below.

This project is essentially asking you to evaluate several different harvesting strategies, and
make a proposal to a fishery about how they can run their business. The primary focus should be
on maintaining some level of harvesting so that the business can sell the fish and make a profit,
but at the same time not harvest too many fish too soon which would cause the fish population
to become extinct.

The relevancy and applicability of this project is enormous. Currently, there are a number
of fisheries operating at a profit while doing significant harm to the species of fish they
harvest. In addition, there are many cases where illegal harvesting occurs and threatens some
species with extinction. If you would like to make your report truly exceptional, research some
of these cases and add them as evidence to your report.

The three models you will evaluate are constant harvesting, proportional harvesting, and periodic
harvesting. In each case, there are advantages and disadvantages that you must discover. Constant
harvesting means that a fixed number of fish are harvested independent of the current population
size.
In proportional harvesting, the number of fish that are harvested is proportional to the current
size of population of fish. In periodic harvesting, the number of fish harvested depends on time.
You
can think of periodic harvesting as seasonal harvesting. More fish are removed at certain times
of the year.

In your 
report, you should include a discussion of the meaning of each variable and parameter and an 
explanation of why the equation is written the way it is.

We have discussed three general approaches that can be employed to study a differential equation: 
Numerical techniques (Euler's Method) yield graphs of approximate solutions, 
geometric or qualitative techniques (slope fields) provide 
predictions of the long-term behavior of the solution and in special cases analytic techniques
(separation of variables and integration) provide 
explicit formulas for the solution. In your report, you should employ as many of these
techniques as is 
appropriate to help understand the models, and you should consider the following three equations.
$P(t)$ is the population of fish at time $t$, measured in thousands of fish. The units of time
can be interpreted in several different ways, depending on the model. This definition of time
will be something you will want to address in your report while describing each model.

Please refer to the table at the end of this assignment for a list of choices for the parameters $k$, 
$N$, $a$, $b$, and $c$. Each row of this table can be considered as a ``harvesting option.'' These
options are meant to provide a starting point for your investigation.

\begin{enumerate}
    \item \textbf{Logistic growth with constant harvesting}
    \[
        \displaystyle\frac{dP}{dt}=kP\left(1-\displaystyle\frac{P}{N}\right) - a
    \]

The equation represents a logistic model of population growth with constant harvesting at 
a rate $a$, measured in thousands of fish per unit time. 
What will happen to the fish population for various initial conditions? You should consider
initial populations both above and below the given carrying capacity, $N$.

    \item \textbf{Logistic growth with proportional harvesting}
    \[
        \displaystyle\frac{dP}{dt}= kP\left(1-\displaystyle\frac{P}{N}\right)
            - bP
    \]

What will
the parameter $b$ represent? What will be the units of $b$? What will happen to the
fish population for various initial conditions?

    \item \textbf{Logistic growth with periodic harvesting}
    \[
        \displaystyle\frac{dP}{dt}= kP\left(1-\displaystyle\frac{P}{N}\right)
            - c\left(1+\sin\left(2\pi t\right)\right)
    \]

What will
the parameter $c$ represent? What will be the units of $c$? What will happen to the
fish population for various initial conditions?

\end{enumerate}

As always, your report should contain an abstract, procedure, and conclusion. The purpose of this
report is to investigate several different fish harvesting options and make a proposal about which
options are the most effective under different circumstances. For example, in the scenario where
the type of fish is abundant, we may want to pursue a strategy that maximizes the number of fish
harvested. In another scenario, the type of fish may be endangered. In which case, we many want to 
take precautions so that even though we continue to harvest, we run absolutely no risk of
species extinction.

You are not necessarily limited to the ten choices listed in the table. These choices are meant
as suggestions about options for a proposal and to help guide your exploration. What this means
is that your proposal is not required to be in this list.

As part of the report, you will need to provide solutions and graphs for a number of different
differential equations. The SAGE worksheet in this project's folder should provide some examples
about how to use SAGE to solve and graph solutions to the equations you need.

\begin{fullwidth}
\begin{table}
\begin{tabular}{cccccc}
\toprule
Choice & $k$ & $N$ & $a$ & $b$ & $c$ \\
\midrule
1 & 0.25 & 4 & 0.16 & 0.16 & 0.10 \\
2 & 0.50 & 2 & 0.21 & 0.21 & 0.10 \\
3 & 0.20 & 5 & 0.21 & 0.21 & 0.30 \\
4 & 0.20 & 5 & 0.16 & 0.16 & 0.25 \\
5 & 0.25 & 4 & 0.09 & 0.16 & 0.21 \\
6 & 0.20 & 5 & 0.09 & 0.09 & 0.18 \\
7 & 0.50 & 2 & 0.16 & 0.16 & 0.22 \\
8 & 0.20 & 5 & 0.24 & 0.24 & 0.33 \\
9 & 0.25 & 4 & 0.21 & 0.21 & 0.12 \\
10 & 0.50 & 2 & 0.09 & 0.09 & 0.15 \\
\bottomrule
\caption{\label{param1} Possible choices for the parameters}
\end{tabular}
\end{table}
\end{fullwidth}

\end{document}