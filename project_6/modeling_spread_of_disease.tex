%%%%%%
%
% PROJECT 6 - MODELING THE SPREAD OF DISEASE, BASIC SIR
%
% filename: modeling_spread_of_disease.tex
% last modified: 2014-10-25
%
%%%%%%%
%
% 
%%%%%%%

\documentclass
[justified,nohyper]
{tufte-handout}

\usepackage{amsmath}


\usepackage{booktabs}
\usepackage{graphicx}
\usepackage{kmath,kerkis} % The order of the packages matters; kmath changes the default text font
\usepackage[T1]{fontenc}


\begin{document}
\section{Advanced Calculus Project 6: Modeling the Spread of Disease}

\newthought{Many human diseases are contagious:} you ``catch'' them from someone 
who is already infected. Contagious diseases are of many kinds. Smallpox, polio, 
and plague are severe and even fatal, while the common cold and the childhood 
diseases of measles, mumps, and rubella are usually relatively mild. Moreover, you 
can catch a cold over and over again, but you get measles only once. A disease 
like measles is said to ``confer immunity'' on someone who recovers from it. Some 
diseases have the potential to affect large segments of a population; the are 
called epidemics (from the Greek words epi, upon + demos, the people.) 
Epidemiology is the scientific study of these diseases.

In this project, you will describe the development of a basic $SIR$-model that
uses three compartments to mathematically group three segments of a population.
This project has an extension project, which we will do later in the year, that
will take a look at the SARS outbreak in Hong Kong in 2003.

\section{Introduction}

An epidemic is a complicated matter, but the dangers posed by contagion and 
especially by the appearance of new and uncontrollable diseases, compel us to 
learn as much as we can about the nature of epidemics. Mathematics offers a very 
special kind of help. First, we can try to draw out of the situation its 
essential features and describe them mathematically. This is calculus as 
language. We substitute an ``ideal'' mathematical world for the real one. This 
mathematical world is called a model. Second, we can use mathematical insights 
and methods to analyze the model. This is calculus as a tool. Any conclusion we 
reach about the model can then be interpreted to tell us something about the 
reality.

The objective of this project is to investigate and understand how the basic 
$SIR$-model is put together. It turns out this will be a system of differential 
equations based on dividing a fixed population into three groups. There will be 
coefficients in these differential equations, and one of your first tasks should 
be to understand where these coefficients come from and how changes to them 
impact the model.

To start, we'll build a model of an epidemic. Its basic purpose is to help us 
understand the way a contagious disease spreads through a population to the point 
where we can even predict what fraction falls ill, and when. Let's suppose the 
disease we want to model is like measles. In particular,

\begin{itemize}
    \item it is mild, so anyone who falls ill eventually recovers;
    \item it confers permanent immunity on every recovered victim.
\end{itemize}

In addition, we will assume that the affected population is large but fixed in 
size and confined to a geographically well-defined region. To have a concrete 
image, you can imagine the elementary school population of a big city.

At any time, that population can be divided into three distinct classes:

\begin{itemize}
    \item Susceptible: those who have never had the illness and can catch it;
    \item Infected: those who currently have the illness and are contagious;
    \item Recovered: those who have already had the illness and are immune.
\end{itemize}

Suppose we let $S(t)$, $I(t)$, and $R(t)$ denote the number of people in
each of these 
three classes, respectively. Of course, the classes are all mixed together 
throughout the population: on a given day, we may find persons who are 
susceptible, infected, and recovered in the same family. For the purpose of 
organizing our thinking, though, we'll represent the whole population as 
separated into three ``compartments,'' each named for the three variables $S$, 
$I$, and $R$ which are all functions of time, $t$.

\section{How fast do individuals move between compartments?} 
We need to understand the rate at which people can move from one compartment to 
the next. In other words, we need to know how fast people get sick and how long 
does it take them to recover.

Our first task will be to model the recovery rate $R'$. We look at the process of 
recovering first, because it's simpler to analyze. An individual caught in the 
epidemic first falls ill and then recovers. Recovery is just a matter of time. In 
particular, someone who catches measles has the infection for about fourteen 
days. So if we look at the entire infected population today, we can expect to 
find some who have been infected less than one day, some who have been infected 
between one and two days, and so on, up to fourteen days. Those in the last group 
will recover today. In the absence of any definite information about the fourteen 
groups, let's assume they are the same size. Then $1/14^{\text{th}}$ of the 
infected population will recover today:

\[
    \text{today the change in the recovered population} = 
    \dfrac{I\;\text{persons}}{14\;\text{days}}
\]

There is nothing special about today, though. $I$ has a value at any time, and this
value is changing every day. Thus 
we can make the same argument about any other day:

\[
    \text{every day the change in the recovered population} = 
    \dfrac{I\;\text{persons}}{14\;\text{days}}
\]

This equation is telling us about $R'$, the rate at which $R$ is changing. We can write it more simply in the form:

\[
    R' = \dfrac{1}{14}I \; \text{persons per day}
\]

Like any differential 
equation, it links different quantities together. In this case, it links $R'$ to 
$I$. The rate equation for $R$ is the first part of our model of the measles 
epidemic.

Notice that the rate equation for $R'$ does indeed give us a tool to predict 
future values of $R$. Suppose that today, 2100 people are infected and 2500 have 
already recovered. Can we say how large the recovered population will be tomorrow 
or the next day? Since $I=2100$:

\[
    R' = \dfrac{1}{14} \times 2100 = 150\;\text{persons per day}
\]

Therefore, we know that 150 people will recover in a single day, and twice as 
many, or 300, will recover in two days. At this rate, the recovered population 
will number 2650 tomorrow and 2800 the next day.

Of course, these calculations assume that the rate $R'$ is constant at 150 
persons per day for the entire two days. Since $R'=I/14$, the rate is the same 
assuming that $I$ holds constant at 2100 persons. If instead, $I$ varies during 
the two days we would have to adjust the value of $R'$ and the future values of 
$R$ as well. In fact, $I$ does vary over time. We will see this when we analyze 
how the infection is transmitted.

It is convention to set $b=1/k$, where $k$ is the number of days it takes to 
recover. The constant $b$ is called the \textbf{recovery coefficient}. With this 
substitution, we have the first of three differential equations that make up
our $SIR$-model.

\[
    R' = bI\;\text{persons per day}
\]

\section{The Rate of Transmission}
Since susceptibles become infected, we need to understand how the flow of 
individuals from the susceptible compartment move to the infected compartment. In 
other words, we need a rate equation for $S'$ -- that will show us how $S$ 
changes as the infection spreads. While $R'$ depends only on $I$, because 
recovery involves only waiting for people to leave the infected population, $S'$ 
will depend on both $S$ and $I$, because transmission involves contact between 
susceptible and infected persons.

Here's a way to model the transmission rate. First, consider a single susceptible 
person on a single day. On average, this person will contact only a small 
fraction, $p$, of the infected population. For example, suppose there are 5000 
infected children, so $I=5000$. We might expect only a couple of them -- let's 
say 2 -- will be in the same classroom with our ``average'' susceptible. So the 
fraction of contacts is $p=2/I = 2/5000= 0.0004$. The 2 contacts themselves can 
be expressed as $2=(2/I) \times I = pI$ contacts per day per susceptible.

To find how many daily contacts the whole susceptible population will have, we can 
just multiply the average number of contacts per susceptible person by the number 
of susceptibles -- this is $pSI$.

Not all contacts will lead to a new infection -- only a certain fraction $q$ do. 
The more contagious the disease, the larger $q$ will be. Since the number of daily 
contacts is $pSI$, we can expect $qpSI$ new infections per day (i.e., to convert 
contact to infections, multiply by $q$). This becomes $aSI$ if we define $a$ to 
be the product $pq$.

Remember, the value of the recovery coefficient depends only on the illness involved. 
It is the same for all populations. By contrast, the value of $a$ depends on the 
general health of a population and the level of social interaction between its 
members. Thus, when two different populations experience the same illness, the 
values of $a$ could be different. One strategy for dealing with an epidemic is to 
alter the value of $a$. Quarantine does this, as you will learn in a later project.

Since each new infection decreases the number of susceptibles, we have the rate 
equation for $S$:

\[
    S' = -aSI\;\text{persons per day}
\]

The minus sign here tells us that $S$ is decreasing (since $S$ and $I$ are 
positive). We call $a$ the \textbf{transmission coefficient}.


\section{Completing the Model}
The final rate equation we need -- the one for $I'$ -- reflects what is already 
clear from the compartment diagram: every loss in $I$ is due to a gain in $R$, 
while every gain in $I$ is due to a loss in $S$.

So the complete model looks like this:

\begin{align*}
    S' &= -aSI \\
    I' &= aSI - bI \\
    R' &= bI \\
\end{align*}


\section{Example Use of the Model}
Your report should contain the details of how a basic SIR-model is put together.
This will look similar to what is in this project description, but should be
written in your own words.

To use the model, we need to decide on the values of the constants that appear
in the system of differential equations. This means we need values for $a$ and
$b$. Once these values are selected, and we indicate some initial values,
we can solve the system to obtain functions
that tell us how $S$, $I$, and $R$ change over time.

The selection of $b$ is based primarily on the disease itself. It is simply
one divided by the number of days it takes to recover. The choice of $a$ is a lot
more involved. For this project, $a$ is going to be a parameter that we manipulate,
but to start off we can think about $a$ in the following way. Suppose that $I=1$
at $t=0$ -- meaning that there is one person who is sick among a population of people.
Suppose that we also know that after one day, a total of three people are sick.
Using this information, can you come up with an estimate for $a$? Your report will
need to describe your reasoning in how $a$ was selected.

\section{Using SAGE to do some heavy lifting}
Let's put some other concrete numbers in place to help you with the first model.
Suppose that $S=762$ and $I=1$ when $t=0$. You might think about this in the following
way. A boarding school has 762 students and one of them is sick. After one day, a
total of three people are sick, so that means that $S=760$ and $I=3$.

Our goal is to find all three functions for $S$, $I$, and $R$. We will use SAGE to
solve this system of differential equations for us. In the project directory will
be a SAGE worksheet that has some example code. Using this code, we can manipulate
the values of $a$ and $b$. We can also change the number of people in our population,
as well as the initial number of sick people.

Study the code in the SAGE worksheet. Using the values described above, develop a
model with an initial population of 760, one infected person at $t=0$, and a recovery
time of two days. To figure out what $a$ should be, you will need to use the
fact that after one day, a total of three people were sick. This may not be the best
way to compute $a$, but it will work for this project.

As always, your report will need an abstract, procedure, and conclusion. The purpose
of your report is to describe the development of a basic $SIR$-model, explain the
meaning of the parameters, and develop an example use of the model using the data
given above.

The resulting graph showing $S$, $I$, and $R$ will be your primary source for
interpreting the results. You will need to comment on what the graph tells you about
how the disease impacts the population.

Possible extensions of the report include changing the initial conditions and
commenting on what happens to the model. For example, if you simply double 
the initial population,
how do the functions for $S$, $I$, and $R$ change? Another extension would be
to consider what happens in the case where the disease does not confer immunity.
In other words, someone can get sick and recover, and then get sick again.

\end{document}