%%%%%%
%
% PROJECT 8 - GLOTTOCHRONOLOGY
%
% filename: glottochronology.tex
% last modified: 2013-12-24
%
%%%%%%%
%
% IN THIS PROJECT, STUDENTS INVESTIGATE HOW CALCULUS CAN BE APPLIED
% TO THE DATING OF LANGUAGES AND THE INTERACTION BETWEEN LANGUAGES.
%
%
% COHEN PAGE 207.
%
%%%%%%%

\documentclass
[justified,nohyper]
{tufte-handout}

\usepackage{booktabs}
\usepackage{graphicx}
\usepackage{kmath,kerkis} % The order of the packages matters; kmath changes the default text font
\usepackage[T1]{fontenc}

\begin{document}
\section{Advanced Calculus Project 8: Glottochronology}

\newthought{The science of dating languages} is called Glottochronology. On June 27, 1995, as a result of a number of research papers published, the New York Times ran a story about the origins of modern languages. At the center of the article was the exciting but controversial hypothesis that all Western, most African, and many Asian languages evolved from a single language as recently as 12,000 years ago. They called the language \textit{Nostratic}. There are no written texts in Nostratic. How on Earth could anyone learn anything about it?

The discovery and analysis of Nostratic involves a number of statistical techniques, but a surprising tool in glottochronology is Calculus. Suppose we're examining the history of a modern language such as Italian, which we know evolved from Latin over 1,000 years ago. We start by drawing up a list of basic words in Latin at a particular date. Let $N(t)$ represent the number of words in the list that are still part of Italian $t$ years after that date. Glottochronologists believe that the rate of change of $N$ is proportional to $N$:

\[
    N' = kN
\]

This is the rate equation that appears in finance, demography, chemistry, and physics, among other disciplines. From our study of differential equations, we know the general form of functions $N(t)$ that will solve this equation describing how the words in a language change over time. What is interesting is when we start to compare the values of $k$ for different languages.

\textbf{Discovery}: Two researchers, C.W. Feng and M. Swadesh, drew up a list of 210 words of Mandarin, as written in A.D. 950, and found that 167 of these were still in use in 1950. Assuming that the development of Mandarin follows the differential equation $N'=kN$, what is $k$? Another group of researchers, working separately from Feng and Swadesh, studied the Italian language and its relation to Latin. Starting from a list of 210 words in Latin, they found that 144 words remain today. If we assume Italian began to split from Latin about 1730 years ago (this number is an estimate -- refer to the questions that appear later in this document), find the value of $k$ for Italian.

It is important to emphasize here that this analysis should work with any collection of words that is suitably random. In the practice of statistics, we are often very concerned about taking a truly random sample from a population. I encourage you to investigate on your own what this entails. You will be required to use random sampling to select 210 words from a Latin dictionary as part of your report.

\textbf{Application}: Out of a list of 212 English words, 124 are also German. It seems reasonable from this that German and English were once the same language. How long ago was that? Provide detailed analysis of how you arrived at your answer. Even better would be to provide some type of external reference that backs up your finding.

\textbf{Your Report}: Your final report should contain an introduction and reflection about the science of dating languages. What is grottochronology and what insights does it provide about the origin and age of languages? How is Calculus a useful tool in studying languages. Make this section of the report your own! Please do not simply copy and paste an existing description from this document or from the web.

Your report should also contain the answers to the following questions.

\begin{enumerate}
  \item When did Italian split off from Latin? You will need to do some serious research to investigate the answer to this question as it relates to your analysis performed earlier.
  \item Get a Latin dictionary. Randomly select 210 Latin words. Now get an Italian dictionary. Count the number of Latin words that are used in Italian. Be generous. The word may have changed slightly, but if the root is there, accept it. Now compute the age of Italian based on your data.
  \item When did English split from German? After you find out, calculate an estimate of the number of words from German that are still present in the English language today. Note: I am not asking you to count them! Doing that would take a long time. Rather, I want you to make use of the mathematical analysis tools we are developing.
  \item The choice of $k$ seems important to the study of languages. Specifically, it would be nice if we could assume that this value is the same for all languages and how they develop. If this were true, it would make the analysis a lot cleaner. Comment on what you think are reasons why such as assumption is valid and why such assumptions might not be valid. This part will also require that you do some research.
\end{enumerate}

\textbf{About Research}: There are several places in this project where you are required to perform research. What does that mean? Several things will be involved. First, any fact or piece of information that you get from an outside source must be cited. I'm not interested in whether this citation is formated in a specific way -- I'm really just concerned with where you looked for information. What this means is that if you want to use some sort of formal citation method in your report, such as MLA or APA, that is fine but is not required. Whatever you choose to do, please be consistent and clear.

Second, there are a number of opportunities for you to copy what I have said into your own report. For example, your report must contain an introduction to the science of glottochronology. Oddly enough, this document contains such an introduction. Please don't copy what I have written or copy what you get from Wikipedia. What you should do is read a variety of sources and from that develop your own understanding and description of what glottochronology is.

Third, part of the report requires you to list 210 Latin words and then find them in an Italian dictionary. Google and some sort of spreadsheet program like Excel can be quite helpful with this. While I do not need to see the list of 210 Latin words in your report, I should see some type of evidence that you performed the investigation.

\end{document}