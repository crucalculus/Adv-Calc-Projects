\documentclass{article}

\usepackage{amsmath}

% GEOMETRY
\usepackage[margin=1.0in]{geometry}
\geometry{letterpaper}

\begin{document}

\begin{center}
\Huge{Writing math in paragraph style}
\end{center}

Originally written by Tim Hsu, San Jose State University with edits by Eric Abell, St. Andrew's Episcopal School.

Consider the following problem.

\begin{quote}
Mr. Chung is paying Grace and Monty to mow his lawn. It takes 4 hours for Grace to mow the lawn by herself, and it takes 5 hours for Monty to mow the lawn by himself. If Monty begins mowing Mr. Chung's lawn at 9am, and Grace joins him at 10am, at what time will they finish mowing the lawn?
\end{quote}

If you've never written math in paragraph style, here's what you might expect would be the answer to this problem:

\begin{eqnarray*}
	\dfrac{1}{5}x + \dfrac{1}{4}(x - 1) &=& 1 \\
	\dfrac{4}{20}x + \dfrac{5}{20}\left(x - \dfrac{1}{4}\right) &=& 1 \\
	\dfrac{9}{20}x &=& \dfrac{5}{4} \\
	x &=& \dfrac{25}{9} \\
\end{eqnarray*}

If you're supposed to be writing in paragraph style, this answer will get no credit. Paragraph-style solutions must be explanations in complete sentences, showing not only your answer, but also the reasoning you used to obtain your answer. This handout explains how to turn something like the above ``answer'' (which is really more like a rough draft) into a finished solution in paragraph style.

\section*{Why write math in paragraph style?}
If you've never done writing in a math class, you may be wondering what the point is. There are two basic reasons why I want you to do this.

\begin{itemize}

\item \textbf{Writing helps you understand.} Look at the above ``answer'' again. Besides the fact that the answer contains no units and doesn't answer the question that was asked (at what time do they finish mowing?), unless you've been doing a lot of these problems, it may be a little tricky trying to figure out where this answer came from. For instance: Why does the equation that you solve involve 1/5 and 1/4 and not just 5 and 4? Why is there a 1 on the right-hand side of the equation? Where did that $(x-1)$ come from?

The thing is, you can generate the above ``answer'' without really understanding what's going on. That is, if you asked the writer of the above ``answer'' what she did, she might well say, ``Well, when you do this kind of a problem, you invert the 4 and the 5, and you subtract 1 sometimes, like when you have $10 - 9 = 1$.'' While that procedure happens to give a useful number in this particular problem, it’s not clear if the writer will be able to handle a slightly different problem. There is a model answer listed at the end of this handout.

\item \textbf{Writing helps you get a job.} As you may already suspect, there may not be many times in your life when you have to figure out how long it takes to mow a lawn using algebra. However, if you ever hold any kind of technical job (which would include anything from engineering to medicine to law and everything in between), at some point you'll probably have to solve a technical problem and write it up in a way that makes sense to someone else (your boss, a co-worker, the general public). The skill of writing on a technical topic in a coherent manner is one of the most valuable things you will get out of this class.

\end{itemize}

\section*{Before the writeup}

Especially if you are working on your homework in a group, here's what you need to do to prepare for the writeup. (Something similar, though less formal, is true if you are doing the homework on your own.)

\textbf{Solve the problem.} Of course, working together, you have to come up with a solution to the assigned problems.

\textbf{Take good notes.} Before you finish your meeting, you should each have some kind of rough draft, or at least some scratchwork, describing the solutions. This rough draft should contain enough information for you to concentrate on writing, and not on re-solving the problem, and should include a list of the variables used and what they mean, relevant Algebra work, sketched versions of tables and graphs, and so on.

\textbf{Does your answer make sense?} To save time later, before you finish your meeting, you should ask yourself the following questions.

\begin{itemize}

\item Are the assumptions behind your work reasonable? In particular, you need to identify the important assumptions you made to solve the problem.

\item Did you answer the question that was asked? For instance, the problem about mowing lawns asks for a particular time of day, so your answer should involve a time of day. Similarly, are the units of your answer correct?

\end{itemize}

\section*{Basics of the writeup}

The basic principle to keep in mind when you write a homework in paragraph style is this: \textbf{Your writeup is meant to be an explanation for someone like you.} More specifically, you should be able to read the homework answer and understand how to solve the problem (and ideally, others like it). Therefore, your answer can't just be a succession of formulas. Your answer must be a clear and concise explanation in complete sentences.

Structurally, a writeup is anchored by its beginning and its end. If you have no idea how to do this, here's a way to start:

\begin{itemize}

\item \textbf{Begin by restating the basic problem.} This can usually be done in a sentence or two. This part of your writeup will be called the abstract.

\item \textbf{Finish by answering the question that was asked.} Make sure you have a clear punchline to your answer, and one that answers the precise question that was asked. This part of your writeup will be called the conclusion.

\end{itemize}

For instance, in our example, you might start your answer with, ``We want to find the time when Monty and Grace will finish mowing the lawn,'' and end with, ``Therefore, they finish working 25/9 hours after 9am, or in other words, a little after 11:46am.''

\section*{The writeup in more detail}

Once you establish the basic framework (restating the problem, concluding sentence), you'll want to fill in the rest of your explanation to match. There are many ways to go about this, but here's one step-by-step procedure to help you get started.

\textbf{Set up your answer, using variables.} After re-stating the problem, one of the first things you should do is to name the important variables you're going to use in your answer, and to describe exactly what each variable means (including the units for each variable, when appropriate). Less important variables may be named later, during the course of the answer.

In our example, after restating the problem, you might next write, ``Let $x$ be the number of hours after 9am that have passed when they finish working, and let $A$ be the area of Mr. Chung's lawn, in acres.''

\textbf{Write in complete sentences and ideas, using connecting phrases.} This is the heart of your answer, so let's discuss this in some detail. The basic idea is that you'll have followed several logical steps to solve the problem. For instance, in our example, your logic might be:

\begin{itemize}

\item To get the speed at which Monty and Grace mow, you divide the size of Mr. Chung's lawn by the number of hours each one takes to mow it.

\item Monty will have worked for $x$ hours when they finish, and Grace will have worked for $(x − 1)$ hours, since she starts an hour later. (That is, unless Monty finishes mowing before or after Grace starts; how do you know that doesn’t happen?)

\item When they finish, they will have mowed a total of $A$ acres. On the other hand, knowing Grace's and Monty's mowing speeds and the amount of time they work, we can figure out how many acres they mow, in terms of $x$, so setting that quantity equal to $A$, we can set up an algebraic equation that we can solve for $x$, and thereby find the time when they stop mowing.

\end{itemize}

Once you understand the chain of logic you're using, you can then turn it into a paragraph or short essay. As with any writing, it's important that you use appropriate connecting phrases to make your answer read better.

For example, the preceding three steps can be written as:

\begin{itemize}

\item ``Since it takes Monty 5 hours to mow the lawn by himself, Monty can mow $A/5$ acres per hour. Similarly, Grace can mow $A/4$ acres per hour.''

\item ``Also, since Monty starts working at 9am, he will have worked for $x$ hours when they finish. Since Grace starts working an hour later, she will have worked for $(x−1)$ hours when they finish.''

\item ``Therefore, since they finish when they have mowed $A$ acres of lawn, we see that (equation). Solving this equation for $x$, we get (algebra, finishing with $x$ = 25/9).''

\end{itemize}

Notice that in our writeup, we've left out some of the mundane algebra involved in the problem (e.g., dividing $A$ by 5 to get Monty's speed). Notice also that when the algebra does show up in our final answer, it's really in the service of some larger idea or complete thought. This is how you should think about algebraic work in a math problem: it's not just a bunch of symbols to be manipulated, it's a series of statements, which are just like regular English sentences, only shorter.

\textbf{Use nice graphs and tables.} This doesn't really apply in our example, but often, graphs and tables will be an important part of your explanation. All graphs and tables must be neat and clear. Furthermore, all graphs must be properly scaled and labelled, especially on their axes. We will learn very specific tools that can generate good looking graphs. I will expect that you use these tools for any graph you include in any report.

\textbf{Putting all of our steps together}, and tidying up a bit, we get the following model answer:

\begin{quote}
We want to find the time when Monty and Grace will finish mowing the lawn. Let $x$ be the number of hours after 9am that have passed when they finish working, and let $A$ be the area of Mr. Chung's lawn, in acres. Now, since it takes Monty 5 hours to mow the lawn by himself, Monty can mow $A/5$ acres per hour; and similarly, Grace can mow $A/4$ acres per hour. Also, since Monty starts working at 9am, he will have worked for $x$ hours when they finish, and since Grace starts working an hour later, she will have worked for $(x − 1)$ hours when they finish. Therefore, since they finish when they have mowed $A$ acres of lawn, we see that:

\[
	\dfrac{A}{5}x + \dfrac{A}{4}(x-1) = A
\]

Solving this equation for $x$, we get
\[
	\dfrac{1}{5}x + \dfrac{1}{4}(x-1) = 1
	\dfrac{9}{20}x = \dfrac{5}{4}
	x = \dfrac{25}{9}
\]

Therefore, they finish working $25/9$ hours after 9am, or in other words, a little after 11:46am.
\end{quote}

\section*{After the writeup}

After you're done writing up the homework, it's an excellent idea to go over it again, possibly to revise it. The kinds of questions you want to ask yourself include:
\begin{enumerate}
\item Are there gaps in the logic?
\item Can you make your answer shorter without losing anything?
\item Are there rough spots in the writing?
\end{enumerate}

Try reading your solution aloud (ideally, to a friend) to see if makes sense. This may seem odd or embarrassing at first, but is actually a very effective way to catch errors, gaps in logic, and nonsensical statements. In fact, many mathematical researchers give lectures on the papers they are writing for much the same reason.

\section*{Computers}

Right now, there are a number of different tools that you can use to make your report. Microsoft Word is probably the first, and perhaps only, option you think of. While Microsoft Word is a fine program and easy to use, it is not what we will be using in this class. Microsoft Word is what you use for English papers that don't include equations and graphs. I realize that Microsoft Word has an Equation Editor, but it is notoriously wrong when typsetting equations as well as being tricky to get your equation to look just the way you want.

For this class you will be using SageMathCloud and \LaTeX \, for all your reports and graphs. At first, this will be difficult for you. Expect frustration and plan ahead so that you have time to finish your report by the deadline. If you stick with it and learn the tools, I personally guarantee that you will never want to go back to Microsoft Word! Your reports will look outstanding and will be something you can be proud of.

\section*{Other tips}

Once you get some idea of how to write math in paragraph style, the following tips may help you write better ones.

\textbf{Be brief and to the point, but include enough detail to make your reasoning clear.} Yes, these two tips contradict each other somewhat. What this means is that you have to learn what level of detail is appropriate for your homework, what's an important point, and what isn't. This will come with experience, so be patient.

\textbf{Useful math cliches.} There are several cliched connecting phrases that are very useful when you write a math homework. Some of them are exhibited in our model answer. Here are a few more:

\begin{quote}
From the information given, we knew $A$. Next, we figured out that $B$ was true, because of $A$. Since $B$ was true, when we looked at the blah blah equation and solved for $x$, we got:
			\begin{center}(algebra)\end{center}
Therefore, we deduced $C$. Finally, because of $C$, we reached our conclusion, $D$.
\end{quote}

As you can see, good math writing doesn't have to be tremendously creative; it just has to make sense, go in the correct order, and flow well.

\textbf{Read the textbook} and whatever other mathematics articles you can. The best place to look for worked examples of good math writing is your textbook. The explanations in the book really are what I want you to be able to produce. (And even better, they're great examples of how to solve the problems.)

\end{document}

