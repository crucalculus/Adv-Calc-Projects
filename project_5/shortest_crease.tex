%%%%%%
%
% PROJECT 5 - SHORTEST CREASE
%
% filename: shortest_crease.tex
% last modified: 2014-7-16
%
%%%%%%%
%
%
%%%%%%%

\documentclass
[justified,nohyper]
{tufte-handout}

\usepackage{amsmath}
\usepackage{amsthm}

\usepackage{booktabs}
\usepackage{graphicx}
\usepackage{kmath,kerkis} % The order of the packages matters; kmath changes the default text font
\usepackage[T1]{fontenc}


\newtheoremstyle{mydef}
{\topsep}{\topsep}%
{}{}%
{\bfseries}{}
{\newline}
{%
  \rule{\textwidth}{0.4pt}\\*%
  \thmname{#1}~\thmnumber{#2}\thmnote{\ -\ #3}.\\*[-1.5ex]%
  \rule{\textwidth}{0.4pt}}%

\theoremstyle{mydef}
\newtheorem{definition}{Definition}

\begin{document}
\section{Advanced Calculus Project 5: The Shortest Crease}
\newthought{You have a sheet of paper} that is 6 units wide and 25 units long, placed so that the short side is facing you. Fold the lower right corner over to touch the left side. Your task is to fold the paper in such a way that the length of the crease is minimized. What is the length of the crease?

As always, you will need an abstract and conclusion for this report. Your abstract will be a simple description of the problem and your conclusion will be the length of the crease. The challenge will be to construct the main part of your report to explain how you arrived at your solution as well as proving that your solution is the shortest.

\end{document}