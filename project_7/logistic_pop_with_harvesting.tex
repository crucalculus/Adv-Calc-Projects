%%%%%%
%
% PROJECT 5 - LOGISTIC POPULATION MODELS WITH HARVESTING
%
% filename: logistic_pop_with_harvesting.tex
% last modified: 2013-11-11
%
%%%%%%%
%
% IN THIS PROJECT, STUDENTS HAVE TO MODEL THE POPULATION WITH A LOGISTIC MODEL
% TAKING INTO CONSIDERATION HARVESTING AT DIFFERENT RATES.
%
%
%%%%%%%
%
% TODO:
% 1. PERHAPS MAKE IT CLEARER THAT I WANT STUDENTS TO INVESTIGATE AND DESCRIBE
%    WHAT WILL HAPPEN IN THE PERIODIC HARVESTING MODEL. HOW OFTEN CAN WE HARVEST?
%    HOW CAN WE HIT A TARGET POPULATION AND MAINTAIN THAT POPULATION? WHAT
%    ABOUT THE DIFFERENCES IN FISH FARMS VERSUS A NORMAL ECOLOGICAL SYSTEM?
%
% 
%%%%%%%

\documentclass
[justified,nohyper]
{tufte-handout}

\usepackage{booktabs}
\usepackage{graphicx}
\usepackage{kmath,kerkis} % The order of the packages matters; kmath changes the default text font
\usepackage[T1]{fontenc}

\begin{document}
\includegraphics[scale=0.3]{fish.png}
\section{Advanced Calculus Project 5: Logistic Population Growth with Harvesting}

\newthought{Logistic Models of} population growth can be very useful in modeling population growth in 
an environment with finite resources. In some cases, we may want to also ``harvest'' part of the 
population at regular intervals. In particular, you should imagine a fish population subject to 
various degrees and types of fishing. The differential equation models are given below. In your 
report, you should include a discussion of the meaning of each variable and parameter and an 
explanation of why the equation is written the way it is.

We have discussed three general approaches that can be employed to study a differential equation: 
Numerical techniques yield graphs of approximate solutions, geometric/qualitative techniques provide 
predictions of the long-term behavior of the solution and in special cases analytic techniques provide 
explicit formulas of the solution. In your report, you should employ as many of these techniques as is 
appropriate to help understand the models, and you should consider the following three equations.
$P(t)$ is the population of fish at time $t$, measured in thousands of fish.

Please refer to the table at the end of this assignment for a list of choices for the parameters $k$, 
$N$, $a_1$, and $a_2$. Each row of this table can be considered as a ``harvesting option.''

\begin{enumerate}
    \item \textbf{Logistic growth with constant harvesting}
    \[
        \displaystyle\frac{dP}{dt}=kP\left(1-\displaystyle\frac{P}{N}\right) - a
    \]

The equation represents a logistic model of population growth with constant harvesting at 
a rate $a$. For $a=a_1$, 
what will happen to the fish population for various initial conditions? (Note: This equation is 
autonomous, so you can take advantage of the special techniques that are available for autonomous 
equations.)

    \item \textbf{Logistic growth with periodic harvesting}
    \[
        \displaystyle\frac{dP}{dt}= kP\left(1-\displaystyle\frac{P}{N}\right)
            - a\left(1+\sin\left(t\right)\right)
    \]

The equation is a non-autonomous equation that considers periodic harvesting. What will
the parameter $a$ represent? If $a=a_1$, what will happen to the
fish population for various initial conditions?

\item Consider the same equation as in Part 2 above, but let $a=a_2$. What will happen 
to the fish population for various initial conditions with this value of $a$?

\end{enumerate}

As always, your report should contain an abstract, procedure, and conclusion. The purpose of this
report is to investigate several different fish harvesting options and make a proposal about which
options are the most effective under different circumstances. For example, in the scenario where
the type of fish is abundant, we may want to pursue a strategy that maximizes the number of fish
harvested. In another scenario, the type of fish may be endangered. In which case, we many want to 
take precautions so that even though we continue to harvest, we run absolutely no risk of
species extinction.

As part of the report, you will need to provide solutions and graphs for a number of different
differential equations. The SAGE worksheet in this project's folder should provide some examples
about how to use SAGE to solve and graph solutions to the equations you need.

\begin{fullwidth}
\begin{table}
\begin{tabular}{ccccc}
\toprule
Choice & $k$ & $N$ & $a_1$ & $a_2$ \\
\midrule
1 & 0.25 & 4 & 0.16 & 0.25 \\
2 & 0.50 & 2 & 0.21 & 0.25 \\
3 & 0.20 & 5 & 0.21 & 0.25 \\
4 & 0.20 & 5 & 0.16 & 0.25 \\
5 & 0.25 & 4 & 0.09 & 0.25 \\
6 & 0.20 & 5 & 0.09 & 0.25 \\
7 & 0.50 & 2 & 0.16 & 0.25 \\
8 & 0.20 & 5 & 0.24 & 0.25 \\
9 & 0.25 & 4 & 0.21 & 0.25 \\
10 & 0.50 & 2 & 0.09 & 0.25 \\
\bottomrule
\caption{\label{param1} Possible choices for the parameters}
\end{tabular}
\end{table}
\end{fullwidth}

\end{document}