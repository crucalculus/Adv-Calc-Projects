% PROJECT 3
%
\documentclass[11pt]{article}
\usepackage{../support_files/advanced_calc}
\begin{document}
%%%%%%  MAKE NO EDITS ABOVE THIS LINE!  %%%%%%%%
\makereporttitle{Project 3}
\yourname{Your name goes here}
\today

\begin{report}
\section*{Abstract}
This is where you will place the Abstract for your project report.

\section*{Procedure}
This is where you will place the Procedure you used for the project. 
Remember that this should be in paragraph style. You can include any 
equations you like by using the following template example:

\[
    P(x) = x^3 + 3x - 1
\]

You can also include equations in-line with sentences like this: 
$P(x)=x^3+3x-1$.

You will need to include graphs, as they are an important part of 
the report. The sage worksheet reminds you how to generate and save 
graphs as PDF documents. This project involves graphing logistic 
equations, and the sage worksheet has several examples to help you.

You may also need to graph the slope field for the differential 
equations used to model world oil production. The sage worksheet 
gives an example.

The line below is what pulls the PDF graph into this document.

In this project, we will use the figure environment to wrap our 
graphs and pictures. This will do three things:
\begin{enumerate}
    \item You will be able to have a subtitle on the graph with a 
    figure number.
    \item The figure can be referenced in the text of your report.
    \item The figure may start to move, depending on the text of 
    your report.
\end{enumerate}

\begin{figure}
\includegraphics[width=10cm]{func_plot.pdf}
\caption{\label{l_func_plot} A Function Plot}
\end{figure}

Here is how we can reference the figure in your text. Please refer 
to Figure \ref{l_func_plot} for a nice looking graph. Notice that 
the figure number is automatically updated. What this means is that 
you can have a report with multiple graphs, all referenced using 
keywords rather than numbers, and when changes are made, the figure 
numbers will update automatically.

Please notice the command that specifies the width. You can set this 
to any dimension you like. I have it set for a width of 10cm. As you 
work on your report, you will want to change this width to make your 
graph look appropriately sized.

Please keep in mind that you can also change how the graph looks in 
the sage worksheet by changing the domain $(x,-2,2)$ to something 
else. In fact, that will probably be what you need to do in order to 
show multiple iterations of the root-finding process.

\section*{Conclusion}
This is where you will place the conclusion of your report. This 
section should be a summary of your results as well as your own 
interpretation of their potential usefulness.

Peak oil is a serious and significant debate that is occurring in 
our world right now. We are very dependent on oil and the idea that 
someday we may run out of oil is something we should consider. I'm 
not concerned with whether you agree or disagree with peak oil. In 
this project, I'm primarily concerned with how you perform the 
mathematical analysis and how you interpret the results. You are not 
writing an argument here, you should focus on presenting the facts 
and interpreting results without bias.

\end{report}

\end{document}

%sagemathcloud={"zoom_width":95}

